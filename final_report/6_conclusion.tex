\chapter{Conclusion}

In this report, we have discussed how we evaluated a
Haskell web framework. We compared two frameworks,
a Haskell framework called Yesod, and a Python
framework called Django. We gave a high-level explanation
of how the two frameworks worked and explained the preparation
needed before development could be started in both frameworks.
We also discussed the process of creating a website in
both frameworks

Once the websites were complete, we performed tests to compare
each framework. These tests were used to: evaluate the performance
of each framework, compare how the language features
of each framework help or hinder web development, and test
the resource efficiency of each framework. After performing
these tests and discussing the results, we wrote a section
that discussed who we would recommend the Yesod framework
to. Overall, Yesod is a production ready framework that is
ready to be used (and has been used), in the real world.

Personally, this project has greatly improved my own academic and
development skills. I gained a lot of knowledge of the Haskell
programming language by getting some hands on experience with
Yesod, a framework that uses a lot of advanced features of Haskell.
By writing this report, my academic
research and writing skills have greatly improved, as I had to
perform a lot of research on existing scientific journals about
Haskell web frameworks.
I have also made some contributions to open source Haskell projects, and
have become more interested in the academic side of Computer
Science.
