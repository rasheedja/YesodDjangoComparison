\chapter{Instructions}

This section will give you instructions on how to run the frameworks on your own Machine.
These instructions are for Ubuntu 16.04. The codebase is included on a CD attached to the
physical copy of this report. I can also give you access to the GitHub repository if you
email me at \texttt{rasheeja@aston.ac.uk}.

\section{Yesod}

Install postgresql 9.5
\lstset{language=Bash}
\begin{lstlisting}
    sudo apt-get update -y
    sudo apt-get upgrade -y
    sudo apt-get -y install postgresql postgresql-9.5 postgresql-client postgresql-common libpq-dev
\end{lstlisting}
Create the database
\begin{lstlisting}
    sudo -u postgres psql
    CREATE USER wire WITH PASSWORD `wire';
    CREATE DATABASE yesod_wire OWNER wire;
    CREATE DATABASE yesod_wire_test OWNER wire;
\end{lstlisting}
\begin{lstlisting}
    cd wire-yesod/yesod/wire
    curl -sSL https://get.haskellstack.org/ | sh
    stack build yesod-bin cabal-install --install-ghc
    stack build
\end{lstlisting}
Run the development webserver
\begin{lstlisting}
    cd wire-yesod/yesod/wire
    stack exec -- yesod devel
\end{lstlisting}
You should now be able to access the site at \texttt{localhost:3000}

\section{Django}

Install postgresql 9.5
\begin{lstlisting}
    sudo apt-get update -y
    sudo apt-get upgrade -y
    sudo apt-get -y install postgresql postgresql-9.5 postgresql-client postgresql-common libpq-dev
\end{lstlisting}
Create the database
\begin{lstlisting}
    sudo -u postgres psql
    CREATE USER wire WITH PASSWORD `wire';
    CREATE DATABASE django_wire OWNER wire;
    CREATE DATABASE django_wire_test OWNER wire;
\end{lstlisting}
Install pip and the necessary python libraries
\begin{lstlisting}
    sudo apt-get -y install python3-pip
    sudo pip3 install --upgrade pip
    sudo pip3 install psycopg2 Django django-bootstrap3 django-bootstrap-breadcrumbs
\end{lstlisting}
Make database migrations and run the development webserver
\begin{lstlisting}
    cd wire-django/django/wire
    python3 manage.py migrate
    python3 manage.py runserver
\end{lstlisting}
You should now be able to access the site at \texttt{localhost:8000}
