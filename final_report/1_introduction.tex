\chapter{Introduction}
\label{chap:Introduction}

The goal of this report is to Evaluate a Haskell Web framework. To do this, two functionally
identical websites were created. One website was created using a Haskell web framework,
and another was created using a Python web framework. A number of tests were then performed
to compare these websites, supporting our evaluation of the Haskell web framework.

One issue that individuals face when looking into Haskell web frameworks is the lack of
detailed comparisons between these frameworks and more traditional frameworks that these
individuals likely have experience in. This report will provide an in-depth look at the
advantages and disadvantages of choosing to use a Haskell Web Framework, and will help
these individuals come to an informed decision on whether or not a Haskell Web Framework is
best for them.

\section{The Chosen Frameworks}

Yesod is a fully featured and modular web framework written in Haskell. Yesod claims to use
features of the Haskell programming language to provide a fast, modular, and type safe web framework. By
choosing Yesod as the web framework for the Haskell language, we will be able to determine
whether or not Haskell's type safety, referential transparency, and lazy evaluation is an advantage
or a disadvantage for web developers.

\begin{quote}
Yesod attempts to ease the web development process by playing to the strengths of the Haskell 
programming language. Haskell’s strong compile-time guarantees of correctness not only encompass 
types; referential transparency ensures that we don’t have any unintended side effects. Pattern 
matching on algebraic data types can help guarantee we’ve accounted for every possible case. 
By building upon Haskell, entire classes of bugs disappear. \parencite[Introduction]{yesodBook}
\end{quote}

Django is a Python Web Framework. Django, like Yesod, is a ``batteries included'' web framework,
``instead of having to open up the language to insert your own power (batteries), you just have
to flick the switch and Django does the rest.'' Django was chosen as the second framework
because it is modular, like Yesod, and the popularity of Python is increasing greatly, being
second only to node.js in growth over the last five years. \parencite{djangoBookReasons}. 
Python is also a dynamically typed language, which helped us evaluate whether or not Haskell's static
type checking actually saves time and effort when developing.

The Django and Yesod web frameworks have a similar set of features. This ensured that we could
make a functionally identical site in both of these frameworks with a similar 
amount of effort. This enabled us to make a fair comparison between Django and Yesod, enabling
us to come to a conclusion on whether a Haskell web framework may be a good choice for a 
developer rather than a more traditional web framework.

\section{The Report}

The rest of this report will discuss any pieces of work similar to this project, the process of
writing the code for both frameworks, including any preparation that had to be done, and a detailed
evaluation of the websites that were produced. The evaluation will compare page load speeds,
the reliability of the websites, comparison of how the frameworks work under load, the ease of
debugging issues, the features included in each framework's test suite, and whether the static typing
and type safety of the Haskell language help or hinder the process of developing a website.
