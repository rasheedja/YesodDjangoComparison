\chapter{Preparation}
\label{chap:Preparation}

In this chapter, we will discuss the work that needed to be done before work could
be started on developing websites in Yesod and Django.

\section{Planning the Website}

Before any work was done, a plan was created that indicated the features the
website should contain to ensure that the features of each framework are able
to be fairly tested and evaluated. The website to be created was a twitter
clone with the following features: a home page, authentication, a profile page,
ability to post a message, ability to post `tagged' messages (any words with
a preceding `\#' turns into a link that leads to a search page),
ability to search for messages and users, and an ability to follow other users.
All features implemented would also have unit tests implemented using the
testing tools available in each framework. After each site is feature complete,
a new feature, the ability to message other users, should be implemented.

Implementing these features allows us to test page load speed by navigation
to certain web pages. Safety can be tested by analysis how each framework deals with custom
user input. We can evaluate how the static type checking and type safety features of 
Haskell affects reliability when compared to dynamic type checking in Python. Testing
all of our implemented features allows us to fairly evaluate the features of the
test suites built into each framework. Implementing a new feature once each
site is feature complete will also allow us to test the maintainability of
each framework.

By planning the website before any work was done, there was a clear vision of
what the website should look like. This allowed development to focus on implementing
the specified features rather than trying to create new features while developing
at the same time, ensuring that functionally identical websites are created in both
frameworks that can be fairly compared and evaluated.

\section{Learning the Frameworks}

Even after the planning was completed for both sites, before any development could
be done, the basics of each framework must be learned. This ensures that code
produced follows the latest standards of each framework, the built-in features
of each framework that may help with development are understood and used appropriately,
and code produced is of a high standard, maintainable, and readable.

\subsection{Learning Django}

Because of previous experience with Python and other object oriented web frameworks, 
Django was learned quickly by going through the official Django tutorials and 
documentation.

The Django tutorials themselves were straight forward for someone who has 
experience in Python and other web frameworks. The tutorials walk you through
installing Django and creating your own app. In Django, an app resides in a project
and is a web application that performs some function. An example of an app could be
a web blogging system. A Django project is a collection of apps and configuration
settings for a particular website. After completing the Django tutorials, work
on the planned website was begun. \parencite{djangoIntroDocs}

\subsection{Learning Yesod}

Coming from an object oriented background, it was not trivial to start using
a functional programming language like Haskell. Before development with the
Yesod framework could be started, the Haskell language had to be learned to
an adequate level.

To help with learning Haskell, the book \citetitle{haskellBook} \parencite{haskellBook}
was used. This book walks the reader through learning the Haskell language beginning
with the fundamentals. Reading through several chapters of the book gives
you a basic understanding of programming with Haskell, allowing you to
start learning Yesod itself and referring to the book to understand more
advanced concepts that you may come across while learning Yesod.

To learn Yesod itself, the book \citetitle{yesodBook} \parencite{yesodBook}, 
written by the person who wrote Yesod, was used. The book goes through all of the
features of the Yesod framework in an easy to understand manner. After
reading the book, development on the planned website using the Yesod
framework began.

\section{The Evaluation}

Once the website was feature complete on both frameworks, a series of experiments
were ran to compare the features that we planned to test during the planning phase.
The raw data of these experiments can be found in the appendix and an evaluation
of these results are discussed in chapter ~\ref{chap:Evaluation}.