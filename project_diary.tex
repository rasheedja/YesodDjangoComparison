\documentclass[a4paper,11pt]{article}
\usepackage[T1]{fontenc}
\usepackage[utf8]{inputenc}
\usepackage{lmodern}
\usepackage{mathptmx}
\usepackage{parskip}

\title{FYP Diary}
\author{Junaid Rasheed}
\date{}

\begin{document}

\maketitle
\tableofcontents
\newpage

\section{Meeting 1 - 3rd October 2017}

\subsection{Meeting Notes:}

\subsubsection{Books}

Real World Haskell\\
Haskell from first principles (haskellbook.com)\\
Web application development with Haskell and Yesod (out of date)

\subsubsection{Frameworks / Tools}

Haskell Servant package\\
Snap is alternative to Yesod\\
ghcjs haskell to js\\
haskell stack tool\\
hackage is like npm. Stack can use hackage.\\
Stackage is like stack on top of hackage\\
Use the latest LTS version of haskell from stackage\\
Atom could be useful with their plugins, compare with plugins available for code\\
ghc-mod available for haskell in atom, helpful when developing\\
ide-haskell, linter\\
There is a Haskell plugin for intellij which may work. Good because I would be familiar with the IDE.

\subsubsection{Comparing the two frameworks}

\begin{itemize}
  \item Maintainability
  \begin{itemize}
    \item Make a change to both
  \end{itemize}
  \item Performance
  \item Scalability - could use tools, hard to do on your own
  \item People say Haskell is easier to write code with, less time debugging, once learnt
  \begin{itemize}
    \item We could test this. How much the type checking helps. The different tools available
    \item Can't use line by line debugging
  \end{itemize}
\end{itemize}

\subsubsection{Plan for next meeting}

Do as much as possible for now\\
Come up with rough project definition form\\
Go through some haskell tutorials, haskellbook.com is recommended

\section{Meeting 2 - 12th October 2017}

\subsection{Meeting Notes:}

Look into getting GHC mod compile on save\\
Get the project proposal doc ready for next week\\
Learn Django and get it installed on the laptop\\
Make a basic page in Django and Haskell

\section{Project Definition Form}

\subsection{14th October 2017}

First draught written up and sent to tutor via email for feedback

\subsection{15th October 2017}

Tutor feedback implemented

\subsection{19th October 2017}

Tutor and I signed form. Form is submitted electronically via Turnitin

\section{Meeting 3 - 19th October 2017}

\subsection{Meeting Notes:}

Carry on with the Haskell Programming from First principles book\\
Have some planning for the twitter clone ready

\section{Meeting 4 - 24th October 2017}

\subsection{Meeting Notes:}

Set up a basic homepage in Yesod and Django. Do this over the weekend.\\
Have a play around with the yesod site that’s provided to see what you can focus on.\\
Carry on with the book\\
Setup Docker/Vagrant if you have time at the end, for instructions on setting up the repo\\
Topics important for yesod
\begin{itemize}
  \item Quasi quotes, provided by yesod
  \item Yesod Typeclass could be useful to know
\end{itemize}

\section{Meeting 5 - 10th November 2017}

\subsection{Meeting Notes:}

I’ve created the homepages in both yesod and django. I’ve used tests in django to test a basic app not related to the project

Next week, I want to ensure both home pages are the same and to create tests in both frameworks. I want to progress more through the yesod and haskell book.
Create User models in both yesod and django and create tests for them.

\section{Meeting 6 - 16th November 2017}

\subsection{Meeting Notes:}

I’ve created the homepages in yesod and django and ensured that they both have the same content and styling.

For django, I have added the functionality to allow users to create accounts and log in. I have added unit tests for this and they all pass.

For yesod, I have added the latest version of jquery and bootstrap to the project. I have tried to complete the user account functionality but I am blocked. I am trying to import yesod-auth-hashdb but cannot figure out how to do it. There is some documentation showing how to edit the cabal file but this is overwritten during the build, I believe the data comes from package.yml. Editing package.yml causes strange errors when I try to build the project but I don’t think I am doing it in the correct manner. Need to figure out how to edit the package.yml, edits would result in errors on my computer.

For next week, I want to fix the weird error and get some tests up.

Things to try to resolve the error, try to reproduce it on normal ubuntu. If you can’t resolve it, report it to yesod.

\section{Meeting 7 - 23rd November 2017}

\subsection{Meeting Notes:}

I’ve resolved the random error we had last week.\\
I’ve imported hashdb and have added functionality for users to create accounts and login on the yesod site.\\
Yesod forms rely on bootstrap 3, so downgraded from bootstrap 4 (beta) to 3.

For next time...\\
I want to figure out how to concatenate a Text data variable in Yesod. Have to figure out how to deal with overloaded strings?\\
Finish the user authentication functionality. Show appropriate messages and add extra validation to the yesod form (unique user and email, min and max length of fields).\\
Create tests for the user authentication functionality.\\
Change the forms on Django to use their form model rather than a HTML form. This will let me compare the pros and cons of Django’s and Yesod’s forms.\\
If there is time, add functionality to allow users to post messages. These messages should be saved in the database so that the user can see all the messages they’ve posted when they log in.

The user post message page should use ajax so when they post a message, the part of the div will just reload rather than the whole page.

\section{Meeting 8 - 14th December 2017}

\subsection{Meeting Notes:}
On the yesod site:\\
Have some tests working\\
Users can post messages, be signed up, see other users messages\\
Have some tests working, this is WIP\\

For next time…\\
Get Django messages working\\
Try to get ajax working on both sites, see https://www.yesodweb.com/blog/2013/02/ajax-with-scaffold\\

Interim report plan
\begin{itemize}
  \item Intro
  \item Explain the choices of yesod and django
  \item Do some initial comparisons of the site
  \item My experiences with developing on both sites, what I found easy and hard on the different frameworks.
  \item Advantages and disadvantages of both frameworks.
\end{itemize}

\end{document}

