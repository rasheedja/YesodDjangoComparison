\documentclass[a4paper,11pt,abstract=on,thesis,twoside]{report}
\usepackage[T1]{fontenc}
\usepackage[utf8]{inputenc}
\usepackage{lmodern}
\usepackage{mathptmx}
\usepackage[toc,page]{appendix}
\usepackage{listings}
\usepackage{color}
\usepackage{pdfpages}
\usepackage{lipsum}
\usepackage{float}
\usepackage{textcomp}
\usepackage{bookmark}
\usepackage[english]{babel}
\usepackage{csquotes}
\usepackage{caption}
\usepackage{subcaption}
\usepackage{fancyhdr}
\usepackage{amsmath}
\usepackage{xcolor}
\usepackage[head=15mm,inner=35mm,outer=20mm]{geometry}
\usepackage[style=apa,citestyle=apa,backend=biber,hyperref=true,apabackref=true]{biblatex}
\usepackage{hyperref}
\usepackage{doi}

\DeclareLanguageMapping{english}{english-apa}

\addbibresource{final_report/bibliography.bib}
% \setkomafont{disposition}{\normalfont\bfseries}

\definecolor{mygreen}{rgb}{0,0.6,0}
\definecolor{mygray}{rgb}{0.5,0.5,0.5}
\definecolor{mymauve}{rgb}{0.58,0,0.82}

\lstset{frame=tb,
	aboveskip=3mm,
	belowskip=3mm,
	backgroundcolor=\color{white},   % choose the background color; you must add \usepackage{color} or \usepackage{xcolor}; should come as last argument
	basicstyle={\footnotesize\ttfamily},    % the size of the fonts that are used for the code
	breaklines=true,                 % sets automatic line breaking
	captionpos=b,                    % sets the caption-position to bottom
	commentstyle=\color{mygreen},    % comment style
	columns=flexible,
	keepspaces=true,                 % keeps spaces in text, useful for keeping indentation of code (possibly needs columns=flexible)
	keywordstyle=\color{blue},       % keyword style
	language=Python,                 % the language of the code
	numbers=left,                    % where to put the line-numbers; possible values are (none, left, right)
	numberstyle=\tiny\color{mygray}, % the style that is used for the line-numbers
	numbersep=0pt,
	rulecolor=\color{black},         % if not set, the frame-color may be changed on line-breaks within not-black text (e.g. comments (green here))
	showspaces=false,                % show spaces everywhere adding particular underscores; it overrides 'showstringspaces'
	showstringspaces=false,          % underline spaces within strings only
	showtabs=false,                  % show tabs within strings adding particular underscores
	stepnumber=1,                    % the step between two line-numbers. If it's 1, each line will be numbered
	stringstyle=\color{mymauve},     % string literal style
	tabsize=2,	                     % sets default tabsize to 2 spaces
	postbreak=\mbox{\textcolor{red}{$\hookrightarrow$}\space}, % line breaks with arrows
}

\renewcommand{\lstlistingname}{Code}
\renewcommand{\lstlistlistingname}{List of Code Blocks}
\raggedbottom

\title{
	{Evaluation of a Haskell Web Framework}\\
	{\large Aston University}\\
	{\large BSC Computer Science}
}
\author{Junaid Ali Rasheed}

\begin{document}
\pagenumbering{roman}


\maketitle

\includepdf[pages=-]{pdf_documents/Final_Year_Project_Definition_Form_Signed.pdf}

\tableofcontents
\listoffigures
\lstlistoflistings

\begin{abstract}
	In this report, we have evaluated a Haskell web framework. We did this by comparing
	two websites, one made in a Haskell web framework called Yesod, and another made
	in a Python web framework in Django. These websites were functionally identical
	and were used to support our evaluation by running various experiments including
	page load speeds and load tests. We also compared the documentation and community
	support available for both frameworks. After evaluating all of our data, we came
	to a conclusion that Yesod is a web framework that is ready to be used in a production
	environment. Yesod is already being used on websites that have millions of users,
	and has outperformed Django in the various tests that we performed. We found that the
	type safety of the Haskell language will catch various bugs that may occur when
	developing a website, but we also acknowledge that some developers prefer the dynamic
	typing of a language like Python. In conclusion, we would recommend Yesod to
	developers who are looking for a new web framework and are either experienced
	in Haskell or are willing to dedicate time to learn the language.
\end{abstract}

\cleardoublepage

\pagenumbering{arabic}
\pagestyle{fancy}

\chapter{Introduction}
\label{chap:Introduction}

In this report, we will compare two web frameworks, one written in Haskell and another, more
popular framework, written in a typical object oriented language. To perform the evaluation,
a functionally identical website was created in both frameworks.

One issue that individuals face when looking into Haskell Web frameworks is the lack of
detailed comparisons between these frameworks and more traditional frameworks that these
individuals likely have experience in. This report will provide an in-depth look at the
advantages and disadvantages of choosing to use a Haskell Web Framework, and will help
these individuals come to an informed decision on whether or not a Haskell Web Framework is
best for them.

\section{The Chosen Frameworks}

Yesod is a fully featured and modular web framework written in Haskell. Yesod claims to use
features of the Haskell language to provide a fast, modular, and type safe web framework. By
choosing Yesod as the web framework for the Haskell language, we will be able to determine
whether or not Haskell's type safety, referential transparency, and lazy compiling is an advantage
or a disadvantage for web developers.

\begin{quote}
Yesod attempts to ease the web development process by playing to the strengths of the Haskell 
programming language. Haskell’s strong compile-time guarantees of correctness not only encompass 
types; referential transparency ensures that we don’t have any unintended side effects. Pattern 
matching on algebraic data types can help guarantee we’ve accounted for every possible case. 
By building upon Haskell, entire classes of bugs disappear. \parencite[Introduction]{yesodBook}
\end{quote}

Django is a Python Web Framework. Django, like Yesod, is a ``batteries included'' web framework,
``instead of having to open up the language to insert your own power (batteries), you just have
to flick the switch and Django does the rest.'' \parencite{djangoBookReasons}. I chose Django
for a number of reasons

\begin{itemize}
  \item ``Batteries included'', like Yesod
  \item Modular, like Yesod
  \item Python is now more common than PHP, second only to Node.js \parencite{djangoBookReasons}
\end{itemize}

The Django and Yesod web frameworks have a similar set of features. This ensured that we could
make a functionally identical site in both of these frameworks with a similar 
amount of effort. This enables us to make a fair comparison between Django and Yesod, enabling
us to come to a conclusion on whether a Haskell web framework may be a good choice for a 
developer rather than a more tradition web framework.

\section{The Report}

The rest of this report will discuss any pieces of work similar to this project, the process of
writing the code for both frameworks, including any preparation that had to be done, and a detailed
evaluation of the websites that were produced. The evaluation will compare page load speeds,
the reliability and maintainability of the websites, the ease of writing new code, the ease of
debugging issues, the features including in each framework's test suite, and whether the static typing
and type safety of the Haskell language help or hinder the process of developing a website.
\chapter{Background}
\label{chap:Background}

In this chapter, we will discuss previous work and research pertaining
to Haskell web frameworks and any real-world sites built using a Haskell
web framework.

\section{Similar Previous Work}

Looking through scientific journals, online articles, and blog posts,
you can find many individuals documenting their experiences and performing
reviews of Haskell Web Frameworks. For example, in the IEEE Internet Computing
journal, \citeauthor{snapFramework} have written an article giving an overview
of Snap. According to the article, snap is a simple web framework written
in Haskell where programming is done at a similar level of abstraction
to Java servlets. The article instructs the reader on how to install Snap, 
walks the reader through some sample code, and shows a quick comparison
between Snap and other major web frameworks. The comparison is a benchmark
of each framework, recording the amount of time it takes for each framework
to respond to a request. The benchmark results can be seen in Figure 
~\ref{fig:snapBenchmark}. \parencite{snapFramework}

Figure ~\ref{fig:snapBenchmark} shows the results of two benchmarks, one where a server
responded to a request by sending the string "pong", and another that records how fast
a server can send a 49 kilobyte image. As you can see in the
graph, for the file benchmark, the Snap framework was faster than all other frameworks.
Snap was only beaten by Node.js in the Pong benchmark when logging was turned on.
Turning logging off dramatically increased the performance of Snap, resulting in
Snap being 55\% faster than Node.js in the Pong benchmark. \parencite{snapFramework}

So, from the research done by \citeauthor{snapFramework}, we can see that Haskell
web frameworks can be significantly faster than more traditional web frameworks.
This is because Haskell uses the Glasgow Haskell Compiler (GHC). GHC compiles
Haskell programs into native machine code, ensuring high performance, especially
for concurrent programs such as web servers. Because of this, any Haskell web framework
that uses GHC, including Yesod, will serve requests faster than most traditional
web frameworks. \parencite{ghcSite}

\begin{figure}[H]
    \centering
    \includegraphics[width=0.7\textwidth]{final_report/pics/snapBenchmark.png}
    \caption{Snap and other frameworks, values are in requests per second \\ \textcopyright{} \citeyear{snapFramework} IEEE}
    \label{fig:snapBenchmark}
\end{figure}

\citeauthor{haskellWebComparison} published a blog post on his website with the
title \citetitle{haskellWebComparison}. In the post, \citeauthor{haskellWebComparison}
provides a quick comparison between some popular Haskell Web Frameworks, including
Yesod and Snap. The comparisons include the installation process of each framework,
the way they handle routing -- i.e. pointing a URL to a piece of code that will 
produce a response, and the quality of documentation for each framework. \parencite{haskellWebComparison}

At the end of his comparison, \citeauthor{haskellWebComparison} found that Yesod
was the best framework for his use case. One of the reasons for his choice was
because of the great documentation available for Yesod. The creator of Yesod
has written a book, \citetitle{haskellBook}, that is very comprehensive. The book
is available for free on the Yesod website. The amount of detail contained in this
book is also one of the reasons the Yesod web framework was chosen for this
project. \parencite{haskellWebComparison,yesodBook}

In \citetitle{beginnerYesod}, \citeauthor{beginnerYesod} discusses his experiences
in using Yesod as a beginner to the Haskell programming language. He mentions
the depth and thoroughness of the Yesod book when learning Yesod. However, when
making an actual website, he came across difficulties when trying to implement
features that required the use of functions not in the book. The author had
to check the documentation of the functions on Hackage, a Haskell package archive.
On Hackage, most functions contain a type signature and normally a one line
description. The author mentions how the type signatures would probably be
enough for experienced Haskell developers to work out how to use a function
but, as a beginner, founding out how a function works using types was much more
difficult. Because of this, the author had to spend a lot of time fixing type
errors. \parencite{beginnerYesod}

When first starting the project, I personally experienced the same issues described
in \citeauthor{beginnerYesod}'s blog post. I found it difficult to understand the type
signatures available on Hackage, resulting in spending a lot of time fixing
unmatched type errors when trying to use functions not documented in the book.
However, as I became more experienced in Yesod and Haskell, these problems
became more and more rare as my understanding of Haskell type signatures increased.

\section{Real World Haskell Sites}

Some readers will be concerned about whether or not a Haskell web framework like
Yesod is ready for production websites. This is a valid concern considering the
relatively small amount of Haskell programmers when compared to mainstream 
programming languages. Readers will be pleased to know, however, that there
are some high traffic sites that are built using the Yesod web framework.

Freckle, previously known as Front Row Education, is an education
platform that provides a service to almost 10 million students \parencite{frontrowName}. 
In 2015, Freckle migrated their site to the Yesod web framework and have been 
using it ever since. \citeauthor{frontrow}, the CTO of Freckle, wrote an article
on his experience of using Yesod for a high traffic website. \parencite{frontrow}

In his article, \citeauthor{frontrow} states that the reason they chose a Haskell
Web Framework was because of the low resource usage and the ability to make
quick iterations that the Haskell language gives you. The article also discusses
how static typing saves time when writing unit tests. The developers at Freckle
did not have to deal with checking for null exceptions, mismatched types, and
other common bugs that are annoying to deal with. Spending less time dealing with
dynamic typing gives developers more time in implementing their features. The
modularity of Yesod also allows Freckle to reuse complex code, reducing potential
mistakes by developers, reducing the amount of code that needs to be written,
and allowing code to be updated quickly without the need to repeat changes. All
of these advantages allow developers to be more efficient and write fewer bugs. \parencite{frontrow}

However, there are some issues that the Freckle team came across during their migration.
Haskell builds are normally quite slow, as all external libraries used have to be
compiled during the build process. \citeauthor{frontrow} mentions that builds took
5-10 minutes on their most powerful machines. The author does mention that the team
could improve their build process by, for example, caching built files. The testing
suite caused problems for the team because when a test fails, they could not determine
which condition caused the failure when a test block has multiple conditions. The issue,
however, was reported to developers behind the testing library and the current
version of the testing suite does not have the issue mentioned in the article. The
lack of documentation for some functions also caused some frustration to the development
team, especially for the more junior developers who could not rely on type signatures. \parencite{frontrow}

When Freckle switched their main API to Yesod, their CPU usage rose to 95\%. This issue
did not occur during testing and profiling, in fact, the Freckle team were the first to
experience this particular issue. This is one issue when using a relatively niche language
like Haskell, you have to be comfortable with the idea that you may be the first person
to experience a particular issue. With other popular frameworks, such as Django, any issue
you discover has most likely been found and fixed by other members of the community.

Despite these issues, Freckle decided to stick with Yesod due to the advantages of the
Haskell compiler and the fact that the issues they experienced with regards to documentation
and build time are improving. And, although the community is small, you can almost
always find help by asking on the StackOverflow or Google Groups pages or by visiting
the \#haskell-beginners IRC channel, an online chat room where developers new to Haskell
can quickly and easily get help from more experienced developers.

\chapter{Preparation}
\label{chap:Preparation}

In this chapter, we will discuss the work that needed to be done before work could
be started on developing websites in Yesod and Django.

\section{Planning the Website}

Before any work was done, a plan was created that indicated the features the
website should contain to ensure that the features of each framework are able
to be fairly tested and evaluated. The website to be created was a twitter
clone with the following features: a home page, authentication, a profile page,
ability to post a message, ability to post `tagged' messages (any words with
a preceding `\#' turns into a link that leads to a search page),
ability to search for messages and users, and an ability to follow other users.
All features implemented would also have unit tests implemented using the
testing tools available in each framework. After each site is feature complete,
a new feature, the ability to message other users, should be implemented.

Implementing these features allows us to test page load speed by navigation
to certain web pages. Safety can be tested by analysis how each framework deals with custom
user input. We can evaluate how the static type checking and type safety features of 
Haskell affects reliability when compared to dynamic type checking in Python. Testing
all of our implemented features allows us to fairly evaluate the features of the
test suites built into each framework. Implementing a new feature once each
site is feature complete will also allow us to test the maintainability of
each framework.

By planning the website before any work was done, there was a clear vision of
what the website should look like. This allowed development to focus on implementing
the specified features rather than trying to create new features while developing
at the same time, ensuring that functionally identical websites are created in both
frameworks that can be fairly compared and evaluated.

\section{Learning the Frameworks}

Even after the planning was completed for both sites, before any development could
be done, the basics of each framework must be learned. This ensures that code
produced follows the latest standards of each framework, the built-in features
of each framework that may help with development are understood and used appropriately,
and code produced is of a high standard, maintainable, and readable.

\subsection{Learning Django}

Because of previous experience with Python and other object oriented web frameworks, 
Django was learned quickly by going through the official Django tutorials and 
documentation.

The Django tutorials themselves were straight forward for someone who has 
experience in Python and other web frameworks. The tutorials walk you through
installing Django and creating your own app. In Django, an app resides in a project
and is a web application that performs some function. An example of an app could be
a web blogging system. A Django project is a collection of apps and configuration
settings for a particular website. After completing the Django tutorials, work
on the planned website was begun. \parencite{djangoIntroDocs}

\subsection{Learning Yesod}

Coming from an object oriented background, it was not trivial to start using
a functional programming language like Haskell. Before development with the
Yesod framework could be started, the Haskell language had to be learned to
an adequate level.

To help with learning Haskell, the book \citetitle{haskellBook} \parencite{haskellBook}
was used. This book walks the reader through learning the Haskell language beginning
with the fundamentals. Reading through several chapters of the book gives
you a basic understanding of programming with Haskell, allowing you to
start learning Yesod itself and referring to the book to understand more
advanced concepts that you may come across while learning Yesod.

To learn Yesod itself, the book \citetitle{yesodBook} \parencite{yesodBook}, 
written by the person who wrote Yesod, was used. The book goes through all of the
features of the Yesod framework in an easy to understand manner. After
reading the book, development on the planned website using the Yesod
framework began.

\section{The Evaluation}

Once the website was feature complete on both frameworks, a series of experiments
were ran to compare the features that we planned to test during the planning phase.
The raw data of these experiments can be found in the appendix and an evaluation
of these results are discussed in chapter ~\ref{chap:Evaluation}.
\chapter{Deliverable}
\label{chap:Deliverable}

\chapter{Evaluation}
\label{chap:Evaluation}
In this chapter, we will discuss the quality of our comparison
of Yesod and Django. We will look at the methods that we used
to test each framework, whether or not these methods could
be improved, any other tests we could have ran, and recommendations
for further comparisons to be made.

\section{The Websites}

The websites that were created were functionally identical. These
websites were usable, fully functional, and responsive.
The websites allow users to create accounts, interact with forms,
have their own profile pages, and search for other users or
messages. These features are used by many websites
in the real-world and by implementing these features, we
were able to realistically evaluate both frameworks.

There is one feature that was discussed in the project plan
but was not included in the final implementation of our websites,
and that was the messaging feature. This feature was planned
to be implemented towards the end of development and would
have evaluated the difficulty of making a large change to both
frameworks. Due to time constrains, it was decided to abandon
this feature and instead, run some measurable tests to quickly
and reliably obtain some concrete data.

\section{Site Hosting}

Both of the websites were hosted on Amazon EC2 servers located
in the same place with identical hardware specifications. This
ensured that the only differing factor in our tests was the
framework being used, giving us fair results. There is, however,
one issue with the servers used, they were very underpowered,
with 1GB of RAM and 1 CPU core. This may have been a limiting
factor when measuring the performance of the frameworks, affecting
our results. It would have been ideal to use a powerful server
for our tests, but choices were limited with no budget available
to host a more powerful server.

\section{Testing Methods}

This section will go through each experiment that was performed
on the frameworks. We will discuss the merits of the test,
the reliability of our results, and any way we could improve
the test.

\subsection{Page Load Speed Tests}

These tests were performed on identical servers and were repeated
three times. The time taken to load the page was taken from Chromium's
developer tools under the network tab. An average was measured and
this value was used for the actual comparison. This ensured that
the results we measured were reliable and accurate. Measuring page
load speeds is also an excellent way of measuring the performance
of a web framework, as keeping page load speeds as low as possible
is very important in order to ensure visitors do not get annoyed at
long load times and leave a website.

The reliability of results could have been improved by performing
more repeats of our experiments. We could have measured the load
time of specific AJAX requests to see if any stand out. Using
servers with different specifications would also have been useful
in order to see how more or less powerful hardware affects page
load speeds.

\subsection{Load Tests}

Load tests were performed using a free service called RedLine13. With
RedLine13, we were able to use a free Amazon EC2 instance to perform
load tests. Unfortunately, these instances were limited to 80 users
so we were not able to perform a large load test over a number of hours.
We also only repeated this experiment three times in order to stay
within Amazon's usage limits.

The results that we obtained from this experiment confirmed what we found in the
Page Load speed experiment, Pages in Yesod load 200-500ms faster than pages in Django.
This experiment also resulted in high page load times of around 5 seconds. This
is most likely because of how weak the hosting servers are. Using
a powerful server would have been very useful to see how each framework
works under very heavy loads.

This experiment also told us that Yesod requests send much less data than Django
due to the way Yesod minimises all static files by default.
Using less data is a very desirable feature for web frameworks, as website
visitors may be on limited plans and would prefer not to use websites
that send a lot of data to their devices.

\subsection{Resource Usage}

This value was only measured once after the page load speed experiments
were complete. Because of this, this value is not very reliable. Repeating
our measurements and calculating an average would definitely increase the
reliability of this result. It would have also been useful to measure this
value every hour for an extended period of time, say, 24 hours. This would
tell us how memory usage varies over time, and would tell us if there's
any bug in the framework that may cause memory leaks.

The value itself did tell us that resource usage is very similar in
both frameworks, with Yesod coming out on top. This is a useful value
for readers to know as some users will be mindful of resource usage when
choosing a web framework for their project.

\subsection{Continuous Integration}

The values we got for this were from Travis CI's web interface. Travis CI
was first integrated with the Yesod repository. It was added to the
Django repository at a later date. This meant that there are much more
build times in the Yesod repository compared to the Django repository.
It would have been useful to have a similar number of builds to allow us
to calculate a reliable average for build times.

The value itself may be useful for developers who are concerned with long
build times, but builds are completed in under five minutes for both
repositories, which is not a significant amount of time. It does let
readers know that testing in both frameworks does not take a significant
amount of time, which would be a concern for developers who develop
in a test driven manner.

\subsection{Debugging}

The simulated errors we performed in this test proved how the Haskell
compiler can save the developer a lot of time when developing a website.
In the first test, the compiler caught a bug where a new message was
being saved using the form data itself rather than the message from the
form data. This same bug was not caught by the Python interpreter. 
In fact, Python automatically converted the form data, which was a map, 
into a string, and stored the value in the database. A test case had
to be added to ensure that this would be caught if a similar mistake
would occur in the future.

The second test, misspelling a variable name, also highlighted how
the Haskell compiler can help developers quickly figure out how
to fix their mistakes. The Haskell compiler printed a message
stating the actual variable name that was misspelt. In this case,
the Python interpreter also printed an appropriate exception message
which mentioned that the variable used was undefined.

\section{Further Comparisons}

A refactor towards the end of development would have been very
useful in our comparison. It would have given us information on
how the Haskell compiler can guide you during a refactor. In
theory, if you start the refactor as a small change and
then compile your program, the Haskell compiler should guide you
on where further changes need to be made. For example, if we
decide to add a new field to the message entity, the Haskell
compiler should tell us where the message entity is being used,
helping us efficiently refactor us program.

Under the free usage limits of Amazon Web Services, available EC2 
servers are not very powerful and the monitoring tools are only 
updated every five minutes. Because of this, only very limited
load testing could be performed, with a maximum of 80 users It
was infeasible to monitor load usage during a test due to the
five minute wait between updates. These tests did tell us the
amount of time it took to load a page and the total amount of
data transferred, but it would have been useful to know how the
web frameworks handle longer load tests. Longer load tests would
have told us how each framework manages resources under load
during extended periods of load. These tests may have identified
issues, such as memory leaks, if they were able to be conducted.
Unfortunately, these tests cannot be performed whilst staying
within Amazon's free usage limits.

\chapter{Conclusion}

In this report, we have discussed how we evaluated a
Haskell web framework. We compared two frameworks,
a Haskell framework called Yesod, and a Python
framework called Django. We gave a high-level explanation
of how the two frameworks worked and explained the preparation
needed before development could be started in both frameworks.
We also discussed the process of creating a website in
both frameworks

Once the websites were complete, we performed tests to compare
each framework. These tests were used to: evaluate the performance
of each framework, compare how the language features
of each framework help or hinder web development, and test
the resource efficiency of each framework. After performing
these tests and discussing the results, we wrote a section
that discussed who we would recommend the Yesod framework
to. Overall, Yesod is a production ready framework that is
ready to be used (and has been used), in the real world.

Personally, this project has greatly improved my own academic and
development skills. I gained a lot of knowledge of the Haskell
programming language by getting some hands on experience with
Yesod, a framework that uses a lot of advanced features of Haskell.
By writing this report, my academic
research and writing skills have greatly improved, as I had to
perform a lot of research on existing scientific journals about
Haskell web frameworks.
I have also made some contributions to open source Haskell projects, and
have become more interested in the academic side of Computer
Science.


\printbibliography[heading=bibintoc,title={References}]

\begin{refsection}
\nocite{*}
\printbibliography[heading=bibintoc,title={Bibliography}]  
\end{refsection}

\begin{appendices}

\chapter{Experiments}
\label{app:Experiments}

All experiments were ran on an Amazon EC2 t2.micro instances. These
are cloud servers with 1GB of ram and 1 CPU core. These servers
were setup with Ubuntu and the tools needed to run the websites.
Both servers were located in the US (East).

\section{Page Load Speed}
\label{sec:pageLoadSpeeds}

Each page of the website was loaded three times with an average being
given. The result recorded is the time it takes to load all the files
on the web page. This
value is taken from the `Finish' value located at the bottom of
the Chromium developer console in the `Network' tab. These
experiments were done with caching turned off. Times were measured
in milliseconds. The experiments also started with a fresh website,
i.e. no users or messages. For redirects, the HTML values is recorded
for the page that started the redirect.

\begin{table}[H]
	\caption{Home Page load speed}
	\begin{center}
		\begin{tabular}{ | l | l | l |}
			\hline
			Run & Yesod Page & Django Page \\
			1 & 504 & 750 \\
			2 & 512 & 754 \\
			3 & 517 & 756 \\
			Average & 511 & 753.33 \\
			\hline
		\end{tabular}
	\end{center}
	\label{tab:pageLoadSpeeds}
\end{table}

\begin{table}[H]
	\caption{Search Page load speed}
	\begin{center}
		\begin{tabular}{ | l | l | l |}
			\hline
			Run & Yesod Page & Django Page \\
			1 & 541 & 739 \\
			2 & 510 & 768 \\
			3 & 501 & 762 \\
			Average & 517.33 & 756.33 \\
			\hline
		\end{tabular}
	\end{center}
	\label{tab:searchLoadSpeeds}
\end{table}

\begin{table}[H]
	\caption{Login Page load speed}
	\begin{center}
		\begin{tabular}{ | l | l | l |}
			\hline
			Run  & Yesod Page & Django Page \\
			1 & 401 & 767 \\
			2 & 413 & 843 \\
			3 & 517 & 854 \\
			Average & 443.67 & 821.33 \\
			\hline
		\end{tabular}
	\end{center}
	\label{tab:loginLoadSpeeds}
\end{table}

\begin{table}[H]
	\caption{Signup Page load speed}
	\begin{center}
		\begin{tabular}{ | l | l | l |}
			\hline
			Run & Yesod Page & Django Page \\
			1 & 509 & 775 \\
			2 & 479 & 739 \\
			3 & 483 & 778 \\
			Average & 490.33 & 764 \\
			\hline
		\end{tabular}
	\end{center}
	\label{tab:signupLoadSpeeds}
\end{table}

\begin{table}[H]
	\caption{Create an account load speed}
	\begin{center}
		\begin{tabular}{ | l | l | l |}
			\hline
			Run & Yesod Page & Django Page \\
			1 & 510 & 728 \\
			2 & 500 & 751 \\
			3 & 503 & 767 \\
			Average & 504.33 & 748.67 \\
			\hline
		\end{tabular}
	\end{center}
	\label{tab:signupCreateLoadSpeeds}
\end{table}

\begin{table}[H]
	\caption{Log in to an account speed}
	\begin{center}
		\begin{tabular}{ | l | l | l |}
			\hline
			Run & Yesod Page & Django Page \\
			1 & 514 & 643 \\
			2 & 570 & 760 \\
			3 & 558 & 765 \\
			Average & 547.33 & 722.67 \\
			\hline
		\end{tabular}
	\end{center}
	\label{tab:loginLoginLoadSpeeds}
\end{table}

\begin{table}[H]
	\caption{Logout load speed speed}
	\begin{center}
		\begin{tabular}{ | l | | l | l |}
			\hline
			Run & Yesod Page & Django Page \\
			1 & 496 & 770 \\
			2 & 525 & 750 \\
			3 & 510 & 762 \\
			Average & 510.33 & 761.33 \\
			\hline
		\end{tabular}
	\end{center}
	\label{tab:logoutLoadSpeeds}
\end{table}

\begin{table}[H]
	\caption{Current user's profile page}
	\begin{center}
		\begin{tabular}{ | l | | l | l |}
			\hline
			Run & Yesod Page & Django Page \\
			1 & 560 & 931 \\
			2 & 660 & 936 \\
			3 & 631 & 923 \\
			Average & 617 & 930 \\
			\hline
		\end{tabular}
	\end{center}
	\label{tab:currentProfileLoadSpeeds}
\end{table}

\begin{table}[H]
	\caption{Creating message `test'}
	\begin{center}
		\begin{tabular}{ | l | l | l |}
			\hline
			Run & Yesod Page & Django Page \\
			1 & 680 & 811 \\
			2 & 667 & 806 \\
			3 & 691 & 929 \\
			Average & 679.33 & 848.67 \\
			\hline
		\end{tabular}
	\end{center}
	\label{tab:createMessageLoadSpeeds}
\end{table}

\begin{table}[H]
	\caption{Other profile page with three messages}
	\begin{center}
		\begin{tabular}{ | l | l | l |}
			\hline
			Run & Yesod Page & Django Page \\
			1 & 670 & 936 \\
			2 & 643 & 931 \\
			3 & 641 & 859 \\
			Average & 651.33 & 908.67 \\
			\hline
		\end{tabular}
	\end{center}
	\label{tab:otherProfileLoadSpeeds}
\end{table}

\begin{table}[H]
	\caption{Search for message `test', three results}
	\begin{center}
		\begin{tabular}{ | l | l | l |}
			\hline
			Run & Yesod Page & Django Page \\
			1 & 514 & 781 \\
			2 & 502 & 773 \\
			3 & 524 & 745 \\
			Average & 513.33 & 766.33 \\
			\hline
		\end{tabular}
	\end{center}
	\label{tab:searchMessageLoadSpeeds}
\end{table}

\begin{table}[H]
	\caption{Search for user `test', five results}
	\begin{center}
		\begin{tabular}{ | l | l | l |}
			\hline
			Run & Yesod Page & Django Page \\
			1 & 524 & 774 \\
			2 & 516 & 748 \\
			3 & 517 & 748 \\
			Average & 519 & 756.67 \\
			\hline
		\end{tabular}
	\end{center}
	\label{tab:searchUserLoadSpeeds}
\end{table}

\section{Resource Usage}

For Yesod, a Keter bundle is deployed on the server. Keter is a deployment
manager written in Haskell and Yesod has built in support for Keter. On
the Django server, nginx is used, a popular open source web server. On the
server, a Django server is created, and nginx uses the Django server to
deal with incoming requests. All values in this section were acquired
through htop.

On the Yesod server, after running all the tests in the previous section,
total RAM usage was 109MB. Keter's main process and sub-processes used about 
83MB of RAM, with 56MB being shared. The Django
server's RAM usage was at 125MB of RAM, with Django and gunicorn using around
65MB of RAM, sharing 19MB, and nginx using 6MB of RAM, sharing 3MB. This gives
a total of 71MB of RAM being used on the Django server, with 22MB of RAM being
shared. Htop sreenshots are included below.

\begin{figure}[H]
	\centering
	\includegraphics[width=1\textwidth]{final_report/pics/yesodIdle.png}
	\caption{Yesod htop output}
	\label{fig:yesodHtop}
\end{figure}

\begin{figure}[H]
	\centering
	\includegraphics[width=1\textwidth]{final_report/pics/djangoIdle.png}
	\caption{Django htop output}
	\label{fig:djangoHtop}
\end{figure}

\section{Continuous Integration Build Times}

Both frameworks are stored on a git repository on GitHub. Whenever there is a
new commit, Travis CI, a continuous integration tools, starts to build
both frameworks and run tests. The tool will tell you whether or not tests
have passed. Django builds take around 2.5 minutes. Yesod builds take around
3.5-4 minutes. It should be noted that the first Yesod build took 32 minutes.
This was because the first build had to compile all the library files in the
Yesod project. Once these are compiled, they are cached, allowing them to
be used for future builds.

\begin{figure}[H]
	\centering
	\includegraphics[width=0.9\textwidth]{final_report/pics/yesodTravis.png}
	\caption{Yesod Travis build times}
	\label{fig:yesodTravis}
\end{figure}

\begin{figure}[H]
	\centering
	\includegraphics[width=0.9\textwidth]{final_report/pics/djangoTravis.png}
	\caption{Django Travis build times}
	\label{fig:djangoTravis}
\end{figure}

\section{Load Tests}

Load tests were ran using RedLine13 and Amazon EC2 servers. Load tests consisted of 80
users loading a specified page. Three tests were conducted. In one test, the home page
was loaded. In the two other tests, the profile page for a user who posted three
messages was loaded. Results can be found in ~\ref{tab:loadTests}

\begin{table}[H]
	\caption{Load Testing Page Load Speeds}
	\begin{center}
		\begin{tabular}{ | l | l | l |}
			\hline
			Page & Yesod (s) & Django (s) \\
			\hline
			Home & 4.96 & 5.54 \\
			Profile & 5.05 & 4.94 \\
			Profile & 4.97 & 5.13 \\
			Average & 4.99 & 5.20 \\
			\hline
		\end{tabular}
	\end{center}
	\label{tab:loadTest1}
\end{table}

\begin{table}[H]
	\caption{Load Testing Data Received}
	\begin{center}
		\begin{tabular}{ | l | l | l |}
			\hline
			Page & Yesod (MB) & Django (MB) \\
			\hline
			Home & 6.28 & 17.99 \\
			Profile & 6.76 & 18.74 \\
			Profile & 6.34 & 17.84 \\
			Average & 6.46 & 18.19 \\
			\hline
		\end{tabular}
	\end{center}
	\label{tab:dataTests}
\end{table}

\begin{figure}[H]
	\centering
	\includegraphics[width=0.9\textwidth]{final_report/pics/yesodLoadTest1.png}
	\caption{Yesod Load Test 1}
	\label{fig:yesodLoadTest1}
\end{figure}

\begin{figure}[H]
	\centering
	\includegraphics[width=0.9\textwidth]{final_report/pics/djangoLoadTest1.png}
	\caption{Django Load Test 1}
	\label{fig:djangoLoadTest1}
\end{figure}

\begin{figure}[H]
	\centering
	\includegraphics[width=0.9\textwidth]{final_report/pics/yesodLoadTest2.png}
	\caption{Yesod Load Test 2}
	\label{fig:yesodLoadTest2}
\end{figure}

\begin{figure}[H]
	\centering
	\includegraphics[width=0.9\textwidth]{final_report/pics/djangoLoadTest2.png}
	\caption{Django Load Test 2}
	\label{fig:djangoLoadTest2}
\end{figure}

\begin{figure}[H]
	\centering
	\includegraphics[width=0.9\textwidth]{final_report/pics/yesodLoadTest3.png}
	\caption{Yesod Load Test 3}
	\label{fig:yesodLoadTest3}
\end{figure}

\begin{figure}[H]
	\centering
	\includegraphics[width=0.9\textwidth]{final_report/pics/djangoLoadTest3.png}
	\caption{Django Load Test 3}
	\label{fig:djangoLoadTest3}
\end{figure}

\subsection{Introducing Realistic Errors}

For these tests, we purposefully made an error that could realistically happen.
After we made the error, we see the error messages that each framework gives us.

\subsubsection{Test 1}
When creating a message, try to save the form results rather than
the message extracted from the results into the database.

Yesod Results: Did not compile, unmatched type error. Tests can't be ran
as site did not compile.
Django Results: No error thrown even when submitting new message form.
Message not created. Database entry added, form data was converted to
string and added to Database. Tests passed because they were checking
the database count. This test was fixed to check the database content
as well.

\lstset{language={Haskell}}
\begin{lstlisting}[caption={Yesod Code Change},label={code:yesodTest1LC}]
	(Entity userId _) <- requireAuth
	((result, _), _) <- runFormPost $ messageForm userId
	case result of
		FormSuccess message -> do
			-- _ <- runDB . insert $ message -- original line 
			_ <- runDB . insert $ result -- new line 
\end{lstlisting}

\lstset{language={Python}}
\begin{lstlisting}[caption={Django Code Change},label={code:djangoTest1LC}]
	form = NewWireForm(request.POST)
	if form.is_valid():
		if request.user.is_authenticated:
			message = form.cleaned_data['message']
			try:
				# Message.objects.create(message_text=message, .. # original line
				Message.objects.create(message_text=form.cleaned_data, .. # changed line 1
				# Message.objects.create(message_text=form, .. # changed line 2, after previous line passed
\end{lstlisting}


\lstset{language={Haskell}}
\begin{lstlisting}[caption={Yesod Exception Message},label={code:yesodTest1Exception}]
	- Couldn't match type `PersistEntityBackend (FormResult Message)'
	with `SqlBackend'
	arising from a use of `insert'
	- In the second argument of `(.)', namely `insert'
	In the expression: runDB . insert
	In a stmt of a 'do' block: _ <- runDB . insert $ result
\end{lstlisting}

\begin{figure}[H]
	\centering
	\includegraphics[width=0.9\textwidth]{final_report/pics/djangoMessageDB.png}
	\caption{Django Message Table Values}
	\label{fig:djangoMessageDBTest1}
\end{figure}

\subsubsection{Test 2}
Simply misspell a variable name. Most IDE tools catch this anyway but
error messages are useful to see. The misspelling was done in the code
that returns recommended users on the profile page. This code returns
data in JSON format.

Yesod: Did not compile, variable not in scope error. Compiler recommended
actual variable.

Django: AJAX request returned  a 500 error. Loading the page shows the
debug output. Exception thrown with message ``name `user' is not defined''.
Tests failed.

\lstset{language={Haskell}}
\begin{lstlisting}[caption={Yesod Code Change},label={code:yesodTest2LC}]
	Entity userId user <- requireAuth
	followers <- runDB $ selectList [FollowFollowerId ==. userId] []
	-- See: https://stackoverflow.com/questions/36727794/haskell-persistent-reusing-selectlist
	let followingIds = map (\(Entity _ (Follow _ followingId)) -> followingId) followers
	users <- runDB $ selectList [UserUsername !=. userUsername user, UserId /<-. followingIds] [LimitTo 5]
	let cleanUsers = map (\(Entity uid (User uname _ _)) -> (object ["id" .= uid, "username" .= uname])) users
	-- returnJson cleanUsers -- original line
	returnJson cleanUser -- new line
\end{lstlisting}

\lstset{language={Python}}
\begin{lstlisting}[caption={Django Code Change},label={code:djangoTest2LC}]
	if request.user.is_authenticated:
	follow_query = Follow.objects.filter(follower_id=request.user)
	users = User.objects.filter().exclude(id=request.user.id).exclude(username=excluded_username)\
		.exclude(followed_user__in=follow_query).values('username')[:5]
	# return JsonResponse(list(users), safe=False)  # original line
	return JsonResponse(list(user), safe=False)  # new line
\end{lstlisting}

\lstset{language={Haskell}}
\begin{lstlisting}[caption={Yesod Exception Message},label={code:yesodTest2Exception}]
	Variable not in scope: cleanUser
	Perhaps you meant `cleanUsers' (line 19)
\end{lstlisting}

\lstset{language={Python}}
\begin{lstlisting}[caption={Django Exception Message},label={code:djangoTest2Exception}]
	NameError at /get-recommended-users/test
	name `user' is not defined
\end{lstlisting}

\chapter{Instructions}

This section will give you instructions on how to run the frameworks on your own Machine.
These instructions are for Ubuntu 16.04. The codebase is included on a CD attached to the
physical copy of this report. I can also give you access to the GitHub repository if you
email me at \texttt{rasheeja@aston.ac.uk}.

\section{Yesod}

Install postgresql 9.5
\lstset{language=Bash}
\begin{lstlisting}
    sudo apt-get update -y
    sudo apt-get upgrade -y
    sudo apt-get -y install postgresql postgresql-9.5 postgresql-client postgresql-common libpq-dev
\end{lstlisting}
Create the database
\begin{lstlisting}
    sudo -u postgres psql
    CREATE USER wire WITH PASSWORD `wire';
    CREATE DATABASE yesod_wire OWNER wire;
    CREATE DATABASE yesod_wire_test OWNER wire;
\end{lstlisting}
\begin{lstlisting}
    cd wire-yesod/yesod/wire
    curl -sSL https://get.haskellstack.org/ | sh
    stack build yesod-bin cabal-install --install-ghc
    stack build
\end{lstlisting}
Run the development webserver
\begin{lstlisting}
    cd wire-yesod/yesod/wire
    stack exec -- yesod devel
\end{lstlisting}
You should now be able to access the site at \texttt{localhost:3000}

\section{Django}

Install postgresql 9.5
\begin{lstlisting}
    sudo apt-get update -y
    sudo apt-get upgrade -y
    sudo apt-get -y install postgresql postgresql-9.5 postgresql-client postgresql-common libpq-dev
\end{lstlisting}
Create the database
\begin{lstlisting}
    sudo -u postgres psql
    CREATE USER wire WITH PASSWORD `wire';
    CREATE DATABASE django_wire OWNER wire;
    CREATE DATABASE django_wire_test OWNER wire;
\end{lstlisting}
Install pip and the necessary python libraries
\begin{lstlisting}
    sudo apt-get -y install python3-pip
    sudo pip3 install --upgrade pip
    sudo pip3 install psycopg2 Django django-bootstrap3 django-bootstrap-breadcrumbs
\end{lstlisting}
Make database migrations and run the development webserver
\begin{lstlisting}
    cd wire-django/django/wire
    python3 manage.py migrate
    python3 manage.py runserver
\end{lstlisting}
You should now be able to access the site at \texttt{localhost:8000}

	
\chapter{Project Diary}
\section{Meeting 1 - 3rd October 2017}

\subsection{Meeting Notes:}

\subsubsection{Books}

Real World Haskell\\
Haskell from first principles (haskellbook.com)\\
Web application development with Haskell and Yesod (out of date)

\subsubsection{Frameworks / Tools}

Haskell Servant package\\
Snap is alternative to Yesod\\
ghcjs haskell to js\\
haskell stack tool\\
hackage is like npm. Stack can use hackage.\\
Stackage is like stack on top of hackage\\
Use the latest LTS version of haskell from stackage\\
Atom could be useful with their plugins, compare with plugins available for code\\
ghc-mod available for haskell in atom, helpful when developing\\
ide-haskell, linter\\
There is a Haskell plugin for intellij which may work. Good because I would be familiar with the IDE.

\subsubsection{Comparing the two frameworks}

\begin{itemize}
  \item Maintainability
  \begin{itemize}
    \item Make a change to both
  \end{itemize}
  \item Performance
  \item Scalability - could use tools, hard to do on your own
  \item People say Haskell is easier to write code with, less time debugging, once learnt
  \begin{itemize}
    \item We could test this. How much the type checking helps. The different tools available
    \item Can't use line by line debugging
  \end{itemize}
\end{itemize}

\subsubsection{Plan for next meeting}

Do as much as possible for now\\
Come up with rough project definition form\\
Go through some haskell tutorials, haskellbook.com is recommended

\section{Meeting 2 - 12th October 2017}

\subsection{Meeting Notes:}

Look into getting GHC mod compile on save\\
Get the project proposal doc ready for next week\\
Learn Django and get it installed on the laptop\\
Make a basic page in Django and Haskell

\section{Project Definition Form}

\subsection{14th October 2017}

First draft written up and sent to tutor via email for feedback

\subsection{15th October 2017}

Tutor feedback implemented

\subsection{19th October 2017}

Tutor and I signed form. Form is submitted electronically via Turnitin

\section{Meeting 3 - 19th October 2017}

\subsection{Meeting Notes:}

Carry on with the Haskell Programming from First principles book\\
Have some planning for the twitter clone ready

\section{Meeting 4 - 24th October 2017}

\subsection{Meeting Notes:}

Set up a basic homepage in Yesod and Django. Do this over the weekend.\\
Have a play around with the yesod site that’s provided to see what you can focus on.\\
Carry on with the book\\
Setup Docker/Vagrant if you have time at the end, for instructions on setting up the repo\\
Topics important for yesod
\begin{itemize}
  \item Quasi quotes, provided by yesod
  \item Yesod Typeclass could be useful to know
\end{itemize}

\section{Meeting 5 - 10th November 2017}

\subsection{Meeting Notes:}

I’ve created the homepages in both yesod and django. I’ve used tests in django to test a basic app not related to the project

Next week, I want to ensure both home pages are the same and to create tests in both frameworks. I want to progress more through the yesod and haskell book.
Create User models in both yesod and django and create tests for them.

\section{Meeting 6 - 16th November 2017}

\subsection{Meeting Notes:}

I’ve created the homepages in yesod and django and ensured that they both have the same content and styling.

For django, I have added the functionality to allow users to create accounts and log in. I have added unit tests for this and they all pass.

For yesod, I have added the latest version of jquery and bootstrap to the project. I have tried to complete the user account functionality but I am blocked. I am trying to import yesod-auth-hashdb but cannot figure out how to do it. There is some documentation showing how to edit the cabal file but this is overwritten during the build, I believe the data comes from package.yml. Editing package.yml causes strange errors when I try to build the project but I don’t think I am doing it in the correct manner. Need to figure out how to edit the package.yml, edits would result in errors on my computer.

For next week, I want to fix the weird error and get some tests up.

Things to try to resolve the error, try to reproduce it on normal ubuntu. If you can’t resolve it, report it to yesod.

\section{Meeting 7 - 23rd November 2017}

\subsection{Meeting Notes:}

I’ve resolved the random error we had last week.\\
I’ve imported hashdb and have added functionality for users to create accounts and login on the yesod site.\\
Yesod forms rely on bootstrap 3, so downgraded from bootstrap 4 (beta) to 3.

For next time...\\
I want to figure out how to concatenate a Text data variable in Yesod. Have to figure out how to deal with overloaded strings?\\
Finish the user authentication functionality. Show appropriate messages and add extra validation to the yesod form (unique user and email, min and max length of fields).\\
Create tests for the user authentication functionality.\\
Change the forms on Django to use their form model rather than a HTML form. This will let me compare the pros and cons of Django’s and Yesod’s forms.\\
If there is time, add functionality to allow users to post messages. These messages should be saved in the database so that the user can see all the messages they’ve posted when they log in.

The user post message page should use ajax so when they post a message, the part of the div will just reload rather than the whole page.

\section{Meeting 8 - 14th December 2017}

\subsection{Meeting Notes:}
On the yesod site:\\
Have some tests working\\
Users can post messages, be signed up, see other users messages\\
Have some tests working, this is WIP

For next time…\\
Get Django messages working\\
Try to get ajax working on both sites, see https://www.yesodweb.com/blog/2013/02/ajax-with-scaffold

Interim report plan
\begin{itemize}
  \item Intro
  \item Explain the choices of yesod and django
  \item Do some initial comparisons of the site
  \item My experiences with developing on both sites, what I found easy and hard on the different frameworks.
  \item Advantages and disadvantages of both frameworks.
\end{itemize}

\section{Meeting 9 - 1st February 2018}

\subsection{Meeting Notes:}
Worked mainly on the Django site. I have the messages working and have began comparing features between two sites such as
\begin{itemize}
  \item The implementation of Handlers/Routes
  \item The way you can pass variables to templates
  \item How Haskell's 'maybe' reduces the number of errors you need to catch
  \item the ways you can implement AJAX in both frameworks
\end{itemize}

In the near future, refactor the messages implementation to use AJAX for retrieval of messages and creating new messages. This refactoring will help compare the ease of modifiability of both of these frameworks.\\
Whenever you come across a difficult error, try to compare the process of debugging in both frameworks.\\
Remember to focus on using different parts of the framework than just implementing new features on the site.\\
Try to resolve the textarea problem. If you can't send a screenshot of the error.\\







\chapter{Ethics Form}
\includepdf[pages=1]{pdf_documents/seas-ethics-student-project-CS-FYP.pdf}

\end{appendices}

\end{document}
